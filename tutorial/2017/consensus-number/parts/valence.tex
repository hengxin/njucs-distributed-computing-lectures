%%%%%%%%%%%%%%%
\begin{frame}{}
  \centerline{\Large \blue{What is a Protocol $\mathcal{P}$?}}

  \vspace{0.60cm}
  \begin{description}
    \item[Protocol] 
      \[
	\mathcal{P} = \set{\text{All executions of this protocol}}
      \]
    \item[Execution] 
      \[
	e = \sigma_0 \xrightarrow{o_1} \sigma_1 \xrightarrow{o_2} \cdots \xrightarrow{o_{n-1}} \sigma_{n}
      \]
    \item[State] 
      \[
	\sigma_i: \text{ states of individual threads} + \text{states of shared objects}
      \]
    \item[Operation] 
      \[
	o_i: \text{ method calls to a shared object}
      \]
  \end{description}
\end{frame}
%%%%%%%%%%%%%%%

%%%%%%%%%%%%%%%
\begin{frame}{}
  \begin{center}
    Modeling $\mathcal{P}$ as a Computation ``Tree'' \\[2pt]
    {\small (Binary Consensus for $2$ Threads)}
  \end{center}
  
  \fignocaption{width = 0.60\textwidth}{figs/state-valent}
\end{frame}
%%%%%%%%%%%%%%%

%%%%%%%%%%%%%%%
\begin{frame}{}
  \begin{theorem}[Bivalent Initial State]
    Every $2$-thread binary consensus protocol has a bivalent initial state.
  \end{theorem}

  \pause
  \[
    (A, B, O): (0,0,\ast) \quad (1,1,\ast) \quad (0,1,\ast) \quad (1,0,\ast)
  \]

  \pause
  \fignocaption{width = 0.40\textwidth}{figs/bivalent-init-proof}
\end{frame}
%%%%%%%%%%%%%%%

%%%%%%%%%%%%%%%
\begin{frame}{}
  \begin{lemma}[Bivalent Initial State]
    Every $n$-thread (binary) consensus protocol has a bivalent initial state.
  \end{lemma}

  \pause
  \vspace{0.30cm}
  \begin{theorem}[Existence of Critical States]
    Every wait-free consensus protocol has a critical state.
  \end{theorem}

  \pause
  \vspace{0.30cm}
  \begin{proof}
    fig here.
  \end{proof}
\end{frame}
%%%%%%%%%%%%%%%

%%%%%%%%%%%%%%%
\begin{frame}{}
\end{frame}
%%%%%%%%%%%%%%%

%%%%%%%%%%%%%%%
\begin{frame}{}
\end{frame}
%%%%%%%%%%%%%%%

%%%%%%%%%%%%%%%
\begin{frame}{}
\end{frame}
%%%%%%%%%%%%%%%

%%%%%%%%%%%%%%%
\begin{frame}{}
\end{frame}
%%%%%%%%%%%%%%%

%%%%%%%%%%%%%%%
\begin{frame}{}
\end{frame}
%%%%%%%%%%%%%%%
