% File: preamble.tex
\usepackage{lmodern}

\usepackage{xeCJK}
\usetheme{CambridgeUS} % try Madrid, Pittsburgh
\usecolortheme{beaver}
\usefonttheme[onlymath]{serif} % try "professionalfonts"

\setbeamertemplate{itemize items}[default]
\setbeamertemplate{enumerate items}[default]

\usepackage{comment}
\usepackage{ulem}

\usepackage{amsmath, amsfonts, latexsym, mathtools, tabu}
\newcommand{\set}[1]{\{#1\}}
\newcommand{\ps}[1]{\mathcal{P}(#1)}
\usepackage{bm}
\DeclareMathOperator*{\argmin}{\arg\!\min}

\ExplSyntaxOn
\bool_new:N \l__xeCJK_listings_letter_bool
\ExplSyntaxOff
\usepackage{listings}

\definecolor{bgcolor}{rgb}{0.95,0.95,0.92}

\lstdefinestyle{CStyle}{
    language = C,
    basicstyle = \ttfamily\bfseries,
    backgroundcolor = \color{bgcolor},   
    keywordstyle = \color{blue},
    stringstyle = \color{teal},
    commentstyle = \color{cyan},
    breakatwhitespace = false,
    breaklines = true,                 
    mathescape = true,
    escapeinside = ||,
    morekeywords = {private, public, final, interface, implements, extends, class, procedure, end, foreach, repeat, until},
    showspaces = false,                
    showstringspaces = false,
    showtabs = false,                  
}

\definecolor{javared}{rgb}{0.6,0,0} % for strings
\definecolor{javagreen}{rgb}{0.25,0.5,0.35} % comments
\definecolor{javapurple}{rgb}{0.5,0,0.35} % keywords
\definecolor{javadocblue}{rgb}{0.25,0.35,0.75} % javadoc
 
\lstdefinestyle{JavaStyle}{
  language=Java,
  basicstyle=\ttfamily\bfseries,
  keywordstyle=\color{javapurple}\bfseries,
  stringstyle=\color{javared},
  commentstyle=\color{javagreen},
  morecomment=[s][\color{javadocblue}]{/**}{*/},
  mathescape = true,
  escapeinside = ||,
  numbers=left,
  numberstyle=\tiny\color{cyan},
  stepnumber=1,
  numbersep = 1pt,
  tabsize=4,
  showspaces=false,
  showstringspaces=false
}

% colors
\newcommand{\red}[1]{\textcolor{red}{#1}}
\newcommand{\redoverlay}[2]{\textcolor<#2>{red}{#1}}
\newcommand{\green}[1]{\textcolor{green}{#1}}
\newcommand{\greenoverlay}[2]{\textcolor<#2>{green}{#1}}
\newcommand{\blue}[1]{\textcolor{blue}{#1}}
\newcommand{\blueoverlay}[2]{\textcolor<#2>{blue}{#1}}
\newcommand{\purple}[1]{\textcolor{purple}{#1}}
\newcommand{\cyan}[1]{\textcolor{cyan}{#1}}
\newcommand{\teal}[1]{\textcolor{teal}{#1}}

% colorded box
\newcommand{\rbox}[1]{\red{\boxed{#1}}}
\newcommand{\gbox}[1]{\green{\boxed{#1}}}
\newcommand{\bbox}[1]{\blue{\boxed{#1}}}
\newcommand{\pbox}[1]{\purple{\boxed{#1}}}

\usepackage{pifont}
\usepackage{wasysym}

\newcommand{\cmark}{\green{\ding{51}}}
\newcommand{\xmark}{\red{\ding{55}}}
%%%%%%%%%%%%%%%%%%%%%%%%%%%%%%%%%%%%%%%%%%%%%%%%%%%%%%%%%%%%%%
% for fig without caption: #1: width/size; #2: fig file
\newcommand{\fignocaption}[2]{
  \begin{figure}[htp]
    \centering
      \includegraphics[#1]{#2}
  \end{figure}
}

\usepackage{tikz}
\usetikzlibrary{positioning, mindmap, shadows}

\newcommand{\push}{\texttt{Push}}
\newcommand{\pop}{\texttt{Pop}}

\newcommand{\titletext}{Consensus Numbers}

\newcommand{\thankyou}{
  \begin{frame}[noframenumbering]{}
    \fignocaption{width = 0.50\textwidth}{figs/thankyou.png}
  \end{frame}
}