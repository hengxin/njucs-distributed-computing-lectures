% file: valence.tex

%%%%%%%%%%%%%%%
\begin{frame}{}
  \begin{center}
    Modeling $\mathcal{P}$ as a \red{Finite Execution ``Tree''} \\[2pt]
    \teal{\footnotesize (Binary Consensus for $2$ Threads)}
  \end{center}
  
  \begin{center}
    \input{tikz/state-valent}
  \end{center}
\end{frame}
%%%%%%%%%%%%%%%

%%%%%%%%%%%%%%%
\begin{frame}{}
  \begin{theorem}[Bivalent Initial State]
    Every \blue{$2$}-thread \blue{binary} consensus protocol has a \red{bivalent initial} state.
  \end{theorem}

  \pause
  \[
    (A, B, O): (0,0,\ast) \quad (1,1,\ast) \quad (0,1,\ast) \quad (1,0,\ast)
  \]

  \pause
  \fig{width = 0.40\textwidth}{figs/bivalent-init-proof}
\end{frame}
%%%%%%%%%%%%%%%

%%%%%%%%%%%%%%%
\begin{frame}{}
  \begin{theorem}[Bivalent Initial State]
    Every \blue{$n$}-thread consensus protocol has a \red{bivalent initial} state.
  \end{theorem}

  \pause
  \vspace{0.30cm}
  \begin{theorem}[Existence of Critical States]
    Every wait-free consensus protocol has a \red{critical} state.
  \end{theorem}

  \pause
  \begin{center}
    {\large \red{By Contradiction: Wait-free!}}
    \input{tikz/critical-state-proof}
  \end{center}
\end{frame}
%%%%%%%%%%%%%%%
