%%%%%%%%%%%%%%%
\begin{frame}{}
  \fignocaption{width = 0.35\textwidth}{figs/fifo-queue}

  \begin{theorem}[Consensus Number of Queue]
    FIFO queues have consensus number $2$.
  \end{theorem}

  \pause
  \begin{proof}
    \begin{columns}
      \column{0.50\textwidth}
	\[ \ge 2 \]
	\uncover<3->{
	  \blue{By protocol for $2$-thread consensus.}
	}
      \column{0.45\textwidth}
	\[ < 3 \]
	\uncover<4->{
	  \blue{By the valency argument.}
	}
    \end{columns}
  \end{proof}
\end{frame}
%%%%%%%%%%%%%%%

%%%%%%%%%%%%%%%
\begin{frame}[fragile]{}
  \begin{theorem}[``Power'' of FIFO Queue]
    The FIFO queue class can solve $2$-thread consensus.
  \end{theorem}

  \pause
  \begin{lstlisting}[style = JavaStyle]
  public class QueueConsensus<T> 
       extends ConsensusProtocol<T> {
    private static final int WIN = 0;
    private static final int LOSE = 1;

    Queue queue;

    // initialize queue with two items
    public QueueConsensus() {
      queue = new Queue();
      queue.enq(WIN);
      queue.enq(LOSE);
    }
  \end{lstlisting}
\end{frame}
%%%%%%%%%%%%%%%

%%%%%%%%%%%%%%%
\begin{frame}[fragile]{}
  \begin{lstlisting}[style = JavaStyle]
  public class QueueConsensus<T> 
       extends ConsensusProtocol<T> {

    // figure out which thread was fist
    public T decide(T val) {
      propose(val);

      int status = queue.deq();
      int i = ThreadID.get();

      if (status == WIN)
        return proposed[i];
      else
        return proposed[1-i];
    }
  }
  \end{lstlisting}
\end{frame}
%%%%%%%%%%%%%%%

%%%%%%%%%%%%%%%
\begin{frame}{}
  \begin{theorem}[``Weakness'' of FIFO Queue]
    The FIFO queue class cannot solve $3$-thread consensus.
  \end{theorem}

  \vspace{0.30cm}
  \begin{proof}
    By the valency argument and case analysis.
  \end{proof}
\end{frame}
%%%%%%%%%%%%%%%
