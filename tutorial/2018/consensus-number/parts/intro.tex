% file: intro.tex

%%%%%%%%%%%%%%%
\begin{frame}{}
  \fig{width = 0.35\textwidth}{figs/survey}
    {\vspace{0.30cm}\centerline{\Large \blue{\bf Concurrent Programming?}}}
\end{frame}
%%%%%%%%%%%%%%%

%%%%%%%%%%%%%%%
\begin{frame}{}
  \fig{width = 0.50\textwidth}{figs/parachuting}
    {\vspace{0.30cm}\centerline{\Large \blue{\bf Synchronization}}}
\end{frame}
%%%%%%%%%%%%%%%

%%%%%%%%%%%%%%%
\begin{frame}{}
  \begin{columns}
    \column{0.45\textwidth}
      \fig{width = 0.50\textwidth}{figs/jcip}
    \column{0.45\textwidth}
      \fig{width = 0.50\textwidth}{figs/cplusplus-concurrency}
  \end{columns}

  \vspace{1.00cm}
  \centerline{\Large \blue{\bf Synchronization Primitives}}

  \pause
  \vspace{0.60cm}
  \begin{columns}
    \column{0.30\textwidth}
      \texttt{\quad\qquad Lock} \\
      \texttt{\quad\qquad Semaphore}
    \column{0.30\textwidth}
      \texttt{\quad BlockingQueue} \\
      \texttt{\quad ConcurrentMap}
    \column{0.30\textwidth}
      \texttt{\qquad Phaser} \\
      \texttt{\qquad Barrier}
  \end{columns}
\end{frame}
%%%%%%%%%%%%%%%

%%%%%%%%%%%%%%%
\begin{frame}[fragile]{}
  \begin{lstlisting}[style = CStyle]
class |\red{AtomicInteger}|: // java.util.concurrent
  get()            // read
  set(int newVal)  // write

  getAndIncrement()
  getAndDecrement()
  getAndSet(int newVal)

  compareAndSet(int expect, int update)|\uncover<2->{\red{\;\;// atomically}}|
  \end{lstlisting}
\end{frame}
%%%%%%%%%%%%%%%

%%%%%%%%%%%%%%%
\begin{frame}[fragile]{}
  \fig{width = 0.30\textwidth}{figs/cas-logo}
  \centerline{\textsf{(compareAndSet \quad compareAndSwap)}}

  \pause
  \vspace{0.50cm}
  \begin{lstlisting}[style = Cstyle]
  |\red{// ''appears to'' be atomic}|
  compareAndSet(int expect, int update) { 
    old = this.val
  
    if (old == expect)
      this.val = update
    
    return old
  }
  \end{lstlisting}
\end{frame}
%%%%%%%%%%%%%%%

%%%%%%%%%%%%%%%
\begin{frame}{}
  \fig{width = 0.30\textwidth}{figs/question-mark}

  \centerline{\Large \blue{(Relative) \red{\it Power} of Various Synchronization Primitives}}
\end{frame}
%%%%%%%%%%%%%%%

%%%%%%%%%%%%%%%
\begin{frame}{}
  \fig{width = 0.60\textwidth}{figs/ruler}

  \centerline{\Large \red{\it Consensus Number:} \blue{$n$-thread consensus problems}}
  
  \pause
  \vspace{0.40cm}
  \begin{columns}
    \column{0.40\textwidth}
      \fig{width = 0.70\textwidth}{figs/hierarchy}
    \column{0.60\textwidth}
      \[
	\text{\Large $1, \quad 2, \quad \cdots, \quad n, \quad \cdots \quad \infty$}
      \]
  \end{columns}
\end{frame}
%%%%%%%%%%%%%%%

%%%%%%%%%%%%%%%
\begin{frame}{}
  \begin{definition}[Consensus Number]
    The \red{\it consensus number} of a class $C$ of synchronization primitive 
    is the largest $n$ for which $C$ \red{\it solves\footnote{\teal{defined later}}} $n$-thread consensus.

    \vspace{0.30cm}
    If no largest $n$ exists, the consensus number is said to be \red{\it infinite}.
  \end{definition}
\end{frame}
%%%%%%%%%%%%%%%

%%%%%%%%%%%%%%%
\begin{frame}{}
  \begin{center}
    \teal{\large The Beautiful Idea of ``Consensus Numbers''} \\
    \teal{(Maurice Herlihy@TOPLAS'1991)}
  \end{center}

  \begin{columns}
    \column{0.50\textwidth}
      \fig{width = 0.90\textwidth}{figs/herlihy}
    \column{0.50\textwidth}
      \fig{width = 0.95\textwidth}{figs/wait-free-sync-paper}
  \end{columns}
\end{frame}
%%%%%%%%%%%%%%%

%%%%%%%%%%%%%%%
\begin{frame}{}
  \begin{theorem}[Consensus Numbers]
    \fig{width = 0.80\textwidth}{figs/consensus-number-table}
  \end{theorem}

  \begin{columns}
    \column{0.60\textwidth}
      \pause
      \begin{description}
	\item[$\infty:$] \red{CAS}
	\item[$1:$] \red{Atomic read/write register}
	\item[$2:$] \textcolor{gray}{Queue, getAndSet}
      \end{description}
    \column{0.40\textwidth}
      \pause
      \fig{width = 0.50\textwidth}{figs/possible-vs-impossible}
  \end{columns}
\end{frame}
%%%%%%%%%%%%%%%
